\documentclass[11pt,a4paper, colorlinks=true, linkcolor=cyan]{moderncv}

% moderncv themes
\moderncvtheme[blue]{classic}
% adjust the page margins
\usepackage[scale=0.8]{geometry}
% required when changes are made to page layout lengths
\recomputelengths

\fancyfoot{} % clear all footer fields
\fancyfoot[LE,RO]{\thepage}
\fancyfoot[RE,LO]{\footnotesize}

% bibliography
\usepackage[sorting=ydnt]{biblatex}
\renewbibmacro*{date}{}
% \renewbibmacro*{date+extrayear}{}
\renewbibmacro*{issue+date}{}
\newcommand*{\bibyear}{}

% This hack is from here: https://tex.stackexchange.com/a/123818/8087
% reformat bib style.
\defbibenvironment{bibliography}
  {\list
     {\iffieldequals{year}{\bibyear}
        {}
        {\printfield{year}%
         \savefield{year}{\bibyear}}}
     {\setlength{\topsep}{0pt}% layout parameters based on moderncvstyleclassic.sty
      \setlength{\labelwidth}{\hintscolumnwidth}%
      \setlength{\labelsep}{\separatorcolumnwidth}%
      \setlength{\itemsep}{\bibitemsep}%
      \leftmargin\labelwidth%
      \advance\leftmargin\labelsep}%
      \sloppy\clubpenalty4000\widowpenalty4000}
  {\endlist}
  {\item}

\addbibresource{pub.bib}

% personal data
\firstname{Dilawar}
\familyname{Singh}
\title{Curriculum Vitae}
\address{Vidyaranyapura, Bangalore}{560097}
\mobile{+919108750725}
\email{dilawar_s@zohomail.in}
\extrainfo{\httplink[ORCID:0000-0002-4645-3211]{https://orcid.org/0000-0002-4645-3211}}
\photo[84pt]{pic_enhanced.jpg}

%----------------------------------------------------------------------------------
%            content
%----------------------------------------------------------------------------------
\begin{document}
\maketitle

\vspace{0mm}

%Section
\section{Info}
\cvdoubleitem{Born}{June 5th, 1985 at Nichalpur (India)}{}{}
\cvdoubleitem{Github}{\small\url{https://github.com/dilawar}\normalsize}{Skype}{\small dilawar\_s}

\vspace{0mm}

%Section

\section{Academic Backgroud}
\cventry{2014-2019}{Ph. D.}{NCBS Bangalore}{}{}{
    Computational Neuroscience,
    Thesis Advisor: \href{https://ncbs.res.in/bhalla}{Prof. Upinder Singh Bhalla}
}

\cventry{2010-2013}{Ph. D.}{IIT Bombay}{}{\textbf{withdrawn}}
{\emph{Partition of large scale digital systems}, Thesis advisor: Prof. Sachin
Patkar \texttt{<patkar@ee.iitb.ac.in>}
}

\cventry{2007-2009}{M. Tech.}{IIT Bombay}{Microelectronics and VLSI}{}
{\emph{Fabrication of micro-electrode arrays for retinal prosthesis}
, Thesis advisor: Prof. Dinesh K. Sharma. \texttt{<dinesh@ee.iitb.ac.in>}}

\cventry{2003-2007}{B. Tech.}{Dr. MGR ERI, Chennai}{}{}
  {Intrumentation and Control Engineering }


% work.
\section{Work Experience}
\cventry{2013-2014}{Research Fellow}{NCBS Bangalore}
{}{}{
    \href{https://github.com/BhallaLab/moose}{MOOSE Simulator}, PI: Prof. Upinder Bhalla
}
\cventry{2009-2010}{Design Engineer}{Kritical Solutions Noida}
{}{}{Embedded systems/Firmware development
Development of firmware for movie-cameras on DINI board with RTOS Multi. Image
stabilization using Kalman filtering. Maintenance of version control system and
servers hosted on Solaris OS.}

\cventry{2016-2018}{GSoC Mentor}{INCF}
{}{}{For 2016, 17, and 18, I was a mentor in Google Summer of Code (GSoC)
    program. I mentored for the organization \href{https://incf.org}{INCF} for
    \href{https://moose.ncbs.res.in}{MOOSE Neural Simulator}. These projects
    involved CUDA/GPU and optimization of solvers.
}

\section{Research Area}
\cventry{Biological Systems}{Robustness, Neural Computation, Memory}
{}{}{}
{
    During my Ph.D., I studied mechanisms which can store information for the lifetime
    of animal. I am very interested in biological systems, especially their
    robustness and probably approximately correct computation and how these
    computations can be replicated in artificial systems.
    Currently I am looking at attention (winner takes all) and habituation (ignoring non-changing
    component of environment) in biological systems and neural mechanisms which
    gives rise to them.
}

\section{Projects}
\cventry{NCBS Hippo}{Content management system and community app}
{}{}{}
{
    \href{NCBS Hippo}{https://ncbs.res.in/hippo} is a RESTful
    website written in PHP7+Codeigniter. It automatically schedules students'
    annual progress seminars using network flow methods (Python+networkx). It also
    manages venue booking, and various talks happening on the campus.
    Repository: \url{https://github.com/dilawar/Hippo}.
    I also wrote an accompanying social Android App using
    \texttt{cordova+Vue+Framework7}. It is available at
    \href{https://play.google.com/store/apps/details?id=com.dilawar.hippo}{Google Play}.
}
\cventry{MOOSE simulator}{Multiscale Object Oriented Simulation Engine}
{}{}{}
{
    During my Ph.D. at NCBS Bangalore, I worked on MOOSE simulator. Specially I
    created CMake based build, CI integration, packaging for PyPI and various
    linux distribution. I also integrated BOOST based ODE solvers to improve the
    efficiency of solvers. I also handle various maintenance related tasks.
    Repository: \url{https://github.com/BhallaLab/moose-core}
}
\cventry{Arduino/PI based behavioural setup}{Animal Behaviour Box}
{}{}{}
{
    An automated behavioural pipeline using Arduino Uno, Point Grey's high speed
    cameras etc.
    Repository: \url{https://github.com/BhallaLab/AnimalBehaviour}.
}
\cventry{Other projects}{}
{}{}{}
{
    My other public projects can be found on \url{https://github.com/dilawar/}.
    Among these, \texttt{CodeSniffer} which checks plagiarism in student's coding
    assignments; a parser of WAV file; eye blink detection (opencv); a tool to
    extract data from old figures are more popular on github.
}



\section{Skills}

\cvline{Languages}{\small C/C++, Python, Haskell, Javascript, VHDL/Verilog/Bluespec, SQL, PHP, Lua, \LaTeX}
\cvline{CAD Tools}{KiCAD, Cadence, Xilinx and Altera tools, Ngspice}
\cvline{Frameworks}{HTML+Vue.js, PHP+Codeigniter, Python+Scipy/Pandas/Matplotlib, pandoc ec. }
\cvline{Software Development}{CMake/Android Studio,
    \href{https://travis-ci.org/dilawar/}{Travis CI}/Jenkins/GITLAB CI
    , Various RPM/DEB on \href{https://build.opensuse.org/users/dilawar}{Open
        Build Service}, Some projects on
        \href{https://pypi.org/user/dilawar/}{PyPI}
}

\nocite{*}
\printbibliography[title={Publications}]

\end{document}
